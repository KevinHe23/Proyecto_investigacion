\documentclass[peerreview]{IEEEtran}
\usepackage{cite}
\usepackage{url}
\usepackage[utf8]{inputenc}
\usepackage{booktabs}
\usepackage{graphicx}
\usepackage[spanish]{babel}
\usepackage{amsmath}
\usepackage{amssymb}
\usepackage{tikz}
\usepackage{hyperref}
\usetikzlibrary{automata,positioning,shapes,arrows,shadows}
\newtheorem{theorem}{\textbf{Teorema}}
\newtheorem{defin}{\textbf{Definición}}
\newtheorem{ejem}{\textbf{Ejemplo}}
\newtheorem{obs}{\textbf{Observación}}
\usepackage{float}
\usepackage{amsmath}
\usepackage{algorithm}
\usepackage{algorithmic}
% traducción
\floatname{algorithm}{Algoritmo}
\renewcommand{\listalgorithmname}{Lista de algoritmos}
\renewcommand{\algorithmicrequire}{\textbf{Entrada:}}
\renewcommand{\algorithmicensure}{\textbf{Salida:}}
\renewcommand{\algorithmicend}{\textbf{fin}}
\renewcommand{\algorithmicif}{\textbf{si}}
\renewcommand{\algorithmicthen}{\textbf{entonces}}
\renewcommand{\algorithmicelse}{\textbf{si no}}
\renewcommand{\algorithmicelsif}{\algorithmicelse,\ \algorithmicif}
\renewcommand{\algorithmicendif}{\algorithmicend\ \algorithmicif}
\renewcommand{\algorithmicfor}{\textbf{para}}
\renewcommand{\algorithmicforall}{\textbf{para todo}}
\renewcommand{\algorithmicdo}{\textbf{hacer}}
\renewcommand{\algorithmicendfor}{\algorithmicend\ \algorithmicfor}
\renewcommand{\algorithmicwhile}{\textbf{mientras}}
\renewcommand{\algorithmicendwhile}{\algorithmicend\ \algorithmicwhile}
\renewcommand{\algorithmicloop}{\textbf{repetir}}
\renewcommand{\algorithmicendloop}{\algorithmicend\ \algorithmicloop}
\renewcommand{\algorithmicrepeat}{\textbf{repetir}}
\renewcommand{\algorithmicuntil}{\textbf{hasta que}}
\renewcommand{\algorithmicprint}{\textbf{imprimir}}
\renewcommand{\algorithmicreturn}{\textbf{devolver}}
\renewcommand{\algorithmictrue}{\textbf{cierto }}
\renewcommand{\algorithmicfalse}{\textbf{falso }}

\begin{document}
\title{Seguridad de la información en la red basada en el sistema de
  criptografía AES}

\author{Kevin Alejandro Herrera\\
  Escuela de Matemática y Ciencias de la Computación\\
  Universidad Nacional Aut\'onoma de Honduras\\
  e-mail: kherrerar@unah.hn
}
\date{\today}

\maketitle
\tableofcontents
\listoffigures
\listoftables

\IEEEpeerreviewmaketitle{}
\begin{abstract}
  Será escrito al terminar el proyecto de investigación
  \bigbreak{}
  \textit{Palabras claves:} AES, criptografía, seguridad de la información,
  cifrado.
\end{abstract}

\section{Introducci\'on}
Será escrito al terminar el proyecto de investigación

\section{Planteamiento del problema}
El avance tecnológico con el pasar del tiempo ha crecido de manera
exponencial, con ello el flujo de información en la red ha aumentado su
volumen, por tanto muchos de estos datos se encuentran vulnerables al no tener
un sistema de seguridad adecuado que pueda resguardar dichos datos, estos
mismos pertenecen a organizaciones, empresas o personas de interés.
\bigbreak{}
A medida avanza la tecnología para bien también avanza para mal y muchos de
estos sistemas quedan obsoletos ante ataques cada vez indefendibles, para ello
se debe implementar un método de seguridad informática; la criptografía en sus
diferentes sistemas ayudará de manera eficiente a proteger esta información.
\bigbreak{}
%%%%%%%%%%%%%%%%%%%%%%%%%%%%%%%%%%%%%%%%%%%%%%%%%%%%%%%%%%%%%%%%%%%%%%
En particular el cifrado simétrico preserva la confidencialidad tanto en las
transmisiones de información como en su almacenamiento, protegiendo los
archivos y evitando que personas ajenas tengan acceso a la información.

\section{Preliminares y Notaci\'on}
\subsection{Entradas y salidas}
La salida y entrada del algoritmo AES consiste en una secuencia de \textit{128
  bits} (dígitos con valores de 0 o 1). Estas secuencias a veces se denominarán
\textit{bloques} y el número de
los bits que contienen se denominarán \textit{longitud}.
\subsection{Bytes}
\begin{defin}[Byte]
  Es una cadena de $8$ \textit{bits}, representados como $\{b_7, b_6, b_5,
    b_4, b_3, b_2, b_1, b_0\}$, estos toman el valor de $0$ o $1$. Es la unidad
  básica de procesamiento del algoritmo AES.\\
\end{defin}

La principal y básica estructura algebraica usada en el sistema AES es el campo
binario de Galois $GF(2)$ (llamados \textit{bits}). Por tanto un \textit{byte}
se puede representar en un campo finito $GF(2^8)$ utilizando una representación
polinomial a través de la fórmula:
\[\sum_{i=0}^{7}b_{i}x^{i}\]

En el algoritmo AES trataremos a los \textit{bytes} en notación hexadecimal,
como se puede ver en el cuadro \ref{tab:HEX}.
\begin{table}[ht]
  \centering
  \begin{tabular}{|c|c|}
    \hline
    \textbf{Cadena de bits} & \textbf{Carácter hexadecimal} \\ \hline
    0000                    & 0                             \\ \hline
    0001                    & 1                             \\ \hline
    0010                    & 2                             \\ \hline
    0011                    & 3                             \\ \hline
    0100                    & 4                             \\ \hline
    0101                    & 5                             \\ \hline
    0110                    & 6                             \\ \hline
    1000                    & 8                             \\ \hline
    1001                    & 9                             \\ \hline
    1010                    & A                             \\ \hline
    1011                    & B                             \\ \hline
    1100                    & C                             \\ \hline
    1101                    & D                             \\ \hline
    1110                    & E                             \\ \hline
    1111                    & F                             \\ \hline
  \end{tabular}
  \caption{Notación hexadecimal}
  \label{tab:HEX}
\end{table}
\subsection{Matriz de estado}
El algoritmo AES divide los datos de entrada en bloques de $4$ palabras de $32$
\textit{bits}, decir $4 \times 32 = 128$ \textit{bits}, se puede ver como una
sucesión de $16$ \textit{bytes}.\\

Para comprender las operaciones que se realizan en AES, representaremos los
$16$ \textit{bytes} como una matriz de $4\times 4$ (cuatro filas por cuatro
columnas) llamada \textbf{matriz de estado}. Un \textit{byte} se representara
por la letra $B$ por tanto nuestra matriz de estado será la siguiente:
\[ \begin{bmatrix}
    B_{15} & B_{14} & B_{13} & B_{12} \\
    B_{11} & B_{10} & B_9    & B_8    \\
    B_7    & B_6    & B_5    & B_4    \\
    B_3    & B_2    & B_1    & B_0
  \end{bmatrix}  \]
\subsection{Operaciones matemáticas}
Todos los bytes en el algoritmo de AES son interpretados como elementos de un
campo finito, estos elementos pueden sumarse y multiplicarse, pero estas
operaciones son diferentes a las que se utilizan con los números usuales. A continuación se detallan.\\
\subsubsection{Suma (o resta)}
La suma y resta se caracteriza por tener el mismo resultado y está dado por el
operador lógico \textit{or} exclusivo (XOR) bit a bit (ver cuadro \ref{tab:XOR}). Esta operación se
representa por el símbolo $\oplus$.

\begin{table}[h]
  \centering
  \begin{tabular}{|c|c|c|}
    \hline
    A & B & A $\oplus$ B \\ \hline
    0 & 0 & 0            \\ \hline
    0 & 1 & 1            \\ \hline
    1 & 0 & 1            \\ \hline
    1 & 1 & 0            \\ \hline
  \end{tabular}
  \caption{\label{tab:XOR}Operación XOR.}
\end{table}

\begin{ejem}
Consideremos $\{0101 1011\}$ y $\{1101 0110\}$
\begin{itemize}
	\item $01011011 \oplus 11010110=10001101$ (Notación binaria)
	
	\item $5B\oplus D6=8D$ (Notación hexadecimal)
	
	\item $(x^6 +x^4 +x^3 +x+1)\oplus (x^7 + x^6 + x^4 + x^2 +x)=x^7 + x^3 +
	x^2 +1$ (Notación polinómica)
\end{itemize}
\end{ejem}


\begin{algorithm}
  \begin{algorithmic}[1]
    \REQUIRE{} $f(x)=\sum_{i=0}^{7}a_{i}x^{i}$, $g(x)=\sum_{i=0}^{7}b_{i}x^{i}
      \in GF(2^8)$
    \ENSURE{}  $h(x)=\sum_{i=0}^{7}c_{i}x^i \in GF(2^8)$
    \FOR{$i=0 \text{ hasta } 7$}
    \STATE{} $c_i  \to (a_i + b_i) \text{ mod } 2$
    \ENDFOR{}
  \end{algorithmic}

  \caption{Algoritmo de la suma en $GF(2^8)$}
  \label{a1}
\end{algorithm}
\subsubsection{Multiplicación}: Se calcula como el residuo de la multiplicación
binaria de dos componentes, teniendo en cuenta la suma y resta en campo finito,
con el polinomio irreducible\footnote{Un polinomio es irreducible si sus únicos
  divisores son el mismo y la unidad.} dado por:
\[m(x)=x^8 + x^4 + x^3 + x +1 \]
Esta operación se denota por el símbolo $\bullet$.

Para multiplicar dos \textit{bytes} se escribirán en forma polinómica, teniendo
en cuenta que la suma es equivalente a una operación XOR. Al realizar la
multiplicación es probable que se tenga un resultado de más de $8$
\textit{bits} por tanto se necesitará hacer modulo de dicho polinomio con el
polinomio irreducible $m(x)$ para así volver a tener $1$ \textit{byte}.

\begin{ejem}
	 Consideremos $\{01010111\}\bullet\{10000011\}$
	\begin{itemize}
		\item $(x^6 +x^4 +x^2 +x^1 + x +1)\bullet(x^7 + x+1)= x^{13} + x^{11} + x^{
			9}+x^{8}+x^{6}+x^5 + x^4 + x^3 +1$
	\end{itemize}
	Evidentemente este nuevo polinomio no contiene $8$ \textit{bits}, entonces:
	\[x^{13} + x^{11} + x^{9}+x^{8}+x^{6}+x^5 + x^4 + x^3 +1 \text{ mod } x^8 + x^4
	+x^3 +x +1\]
	\[=x^7 + x^6 +1\] que corresponde a $11000001$.
\end{ejem}
\begin{algorithm}
  \begin{algorithmic}[1]
    \REQUIRE{} $f(x)=\sum_{i=0}^{7}a_{i}x^{i}$, $g(x)=\sum_{i=0}^{7}b_{i}x^{i}
      \in GF(2^8)$
    \ENSURE{}  $h(x)=\sum_{i=0}^{7}c_{i}x^i \in GF(2^8)$
    \STATE{} $c \leftarrow 0$
    \FOR{$i=0 \text{ hasta } 7$}
    \STATE{} $c_i  \leftarrow b_i f(x) + c$
    \STATE{} $f(x) \leftarrow f(x)(x^i) \text{ mod } m(x)$
    \ENDFOR{}
  \end{algorithmic}

  \caption{Algoritmo de la multiplicación en $GF(2^8)$}
  \label{a2}
\end{algorithm}

%%%%%%%%%%%%%%%%%%%%%%%%%%%%%%%%%%%%%%%%%%%%%%%%%%%%%%%%%%%%%%%%%%%%%%

\section{Seguridad Informática}
\begin{defin}[Seguridad de la información]
  Conjunto de medidas preventivas de las organizaciones  que permiten resguardar
  y proteger la información buscando mantener la confidencialidad, disponibilidad
  e integridad de la misma.
\end{defin}
\bigbreak
Esta seguridad busca la preservación de tres pilares:
\begin{itemize}
  \item La \textit{confidencialidad} es el principio que garantiza que la
        información solo puede ser accedida por las personas que tienen autorización.
  \item La \textit{integridad} es que la información sólo puede ser creada y
        modificada por quien esté autorizado a hacerlo.
  \item La \textit{disponibilidad} es que la información debe ser accesible
        para su consulta o modificación cuando se requiera.
\end{itemize}
\bigbreak
Estos tres conceptos forman lo que a menudo se denomina la triada CIA. Los tres
conceptos encarnan los objetivos fundamentales de seguridad tanto para los
datos como para la información y los servicios informáticos.
\begin{figure}[htbp]
  \centering
  \includegraphics[width=5.5cm]{figuras/cia_tirada.png}
  \caption{Tríada de la CIA}
  \label{fig: 1}
\end{figure}
\bigbreak
\begin{defin}[Seguridad informática]
  Conjunto de políticas y mecanismos que nos permiten garantizar la
  confidencialidad, la integridad y la disponibilidad de los recursos en un
  sistema.
\end{defin}
\bigbreak
\textbf{\textit{Observación:}} El concepto de \textit{seguridad de la
  información} no debe ser confundido con el de \textit{seguridad informática},
ya que este último sólo se encarga de la seguridad en el medio informático. Sin
embargo, la información puede encontrarse en diferentes medios o formas, y no
solo en medios informáticos. Es decir que el primer concepto es mas amplio que
el segundo.
%%%%%%%%%%%%%%%%%%%%%%%%%%%%%%%%%%%%%%%%%%%%%%%%%%%%%%%%%%%%%%%%%%%%%%
\section{Historia de la criptografía}
La palabra \textit{criptografía} proviene en un sentido etimológico  del griego
Kriptos = ocultar, Graphos = escritura, lo que significaría escritura oculta.\\

En su clasificación dentro de las ciencias, la \textit{criptografía} proviene
de una rama de las matemáticas como se muestra en la figura \ref{fig: 2} que
fue iniciada por el matemático Claude Shannon \footnote{Claude Elwood Shannon
  fue un matemático, ingeniero eléctrico y criptógrafo estadounidense recordado
  como \textit{el padre de la teoría de la información}}. en 1948, denominada:
Teoría de la información.\\

\begin{figure}[h]
  \centering
  \includegraphics[scale=0.4]{figuras/origen_criptografía.PNG}
  \caption{Origen de la criptografía.}
  \label{fig: 2}
\end{figure}

\begin{defin}[Criptografía]
  Ciencia encargada de diseñar funciones o dispositivos, capaces de transformar
  mensajes legibles a mensajes cifrados de tal manera que esta transformación
  (cifrar) y su transformación inversa (descifrar) sólo pueden ser factibles con
  el conocimiento de una o mas llaves.\\
  
\end{defin}

\begin{defin}[Criptoanálisis]
	Ciencia que estudia los métodos que se utilizan para descifrar mensajes sin necesidad de tener la llave con la que dicho mensaje fue cifrado.\\
	
\end{defin}
La \textit{criptografía} se puede clasificar en dos: \begin{itemize}
  \item La criptografía clásica
  \item La criptografía moderna
\end{itemize}


%%%%%%%%%%%%%%%%%%%%%%%%%%%%%%%%%%%%%%%%%%%%%%%%%%%%%%%%%%%%%%%%%%%%%%
\section{Advanced Encryption Standard (AES)}

\subsection{Estructura del algoritmo}
La estructura de este algoritmo consiste en una serie de rondas donde se
realizan un conjunto de cuatro transformaciones orientadas a \textit{bytes}, el
número de rondas depende de la longitud de la llave.

Vamos a asumir que la longitud de la llave escogida es de 128 \textit{bits} ya
que la longitud de la llave no afecta a las diversas trasformaciones que
realiza el cifrado.
\begin{table}[h]
  \centering
  \resizebox{9cm}{!}{
    \begin{tabular}{l|c|c|c|}
      \cline{2-4}
                                                                         & \begin{tabular}[c]{@{}c@{}}\textbf{Longitud de
                                                                             llave} \\ \textit{(Nk words)}\end{tabular}         &
      \begin{tabular}[c]{@{}c@{}}\textbf{Tamaño de bloque} \\ \textit{(Nb
          words)}\end{tabular} & \begin{tabular}[c]{@{}c@{}}\textbf{Número de rondas} \\
                               \textit{(Nr)}\end{tabular}                                               \\ \hline
      \multicolumn{1}{|l|}{AES-128}                                      & 4
                                                                         & 4
                                                                         & 10
      \\ \hline
      \multicolumn{1}{|l|}{AES-192}                                      & 6
                                                                         & 4
                                                                         & 12
      \\ \hline
      \multicolumn{1}{|l|}{AES-256}                                      & 8
                                                                         & 4
                                                                         & 14
      \\ \hline
    \end{tabular}
  }
  \caption{Número de rondas en función de la longitud de la llave}
  \label{tab:cuadro3}
\end{table}

Básicamente se aplican cuatro trasformaciones a la matriz de estado durante el
número determinado de rondas, las cuales se detallaran a continuación.
\subsection{Cifrado AES}
Para definir este proceso, vamos a asumir que la longitud de la llave es de 128 \textit{bits}, ya que la longitud de la llave no afecta a las trasformaciones que realiza el cifrado.\\ 
En el proceso de cifrado se aplican cuatro transformaciones a la matriz de estado inicial, dichas transformaciones son las siguientes:\\

\subsubsection{SubBytes()}
Esta operación consiste en una transformación no lineal, donde se realiza una
sustitución de cada byte por otro byte establecido en la caja de sustitución
(S-box) ver cuadro \ref{tab:cuadro4} .\\

\begin{figure}[h!]
  \includegraphics[scale=0.1]{figuras/SubBytes.png}
  \centering
  \caption{Transformación SubBytes}
  \label{fig: 3}
\end{figure}

AES define una matriz de valores de \textit{bytes} de $16 \times 16$ llamada
S-Box, que contiene una permutación de todos los 256 valores de 8 \textit{bits}
posibles.
\begin{table}[ht]
  \centering
  \resizebox{10cm}{!}{
    \begin{tabular}{|c|c|c|c|c|c|c|c|c|c|c|c|c|c|c|c|c|}
      \hline
      -          & \textbf{x0} & \textbf{x1} & \textbf{x2} & \textbf{x3} & \textbf{x4} &
      \textbf{x5} & \textbf{x6} & \textbf{x7} & \textbf{x8} & \textbf{x9} & \textbf{xA} &
      \textbf{xB} & \textbf{xC} & \textbf{xD} & \textbf{xE} & \textbf{xF}                \\ \hline
      \textbf{0x} & 63         & 7E         & 77         & 7B         & F2         &
      6B         & 6F         & C5         & 30         & 01         & 67         &
      2B         & FE         & D7         & AB         & 76                        \\ \hline
      \textbf{1x} & CA         & 82         & C9         & 7D         & FA         &
      59         & 47         & F0         & AD         & D4         & A2         &
      AF         & 9C         & A4         & 72         & C0                        \\ \hline
      \textbf{2x} & B7         & FD         & 93         & 26         & 36         &
      3F         & F7         & CC         & 34         & A5         & E5         &
      F1         & 71         & D8         & 31         & 15                        \\ \hline
      \textbf{3x} & 04         & C7         & 23         & C3         & 18         &
      96         & 05         & 9A         & 07         & 12         & 80         &
      E2         & EB         & 27         & B2         & 75                        \\ \hline
      \textbf{4x} & 09         & 83         & 2C         & 1A         & 1B         &
      6E         & 5A         & A0         & 52         & 3B         & D6         &
      B3         & 29         & E3         & 2F         & 84                        \\ \hline
      \textbf{5x} & 53         & D1         & 00         & ED         & 20         &
      FC         & B1         & 5B         & 6A         & CB         & BE         &
      39         & 4A         & 4C         & 58         & CF                        \\ \hline
      \textbf{6x} & D0         & EF         & AA         & FB         & 43         &
      4D         & 33         & 85         & 45         & F9         & 02         &
      7F         & 50         & 3C         & 9F         & A8                        \\ \hline
      \textbf{7x} & 51         & A3         & 40         & 8F         & 92         &
      9D         & 38         & F5         & BC         & B6         & DA         &
      21         & 10         & FF         & F3         & D2                        \\ \hline
      \textbf{8x} & CD         & 0C         & 13         & EC         & 5F         &
      97         & 44         & 17         & C4         & A7         & 7E         &
      3D         & 64         & 5D         & 19         & 73                        \\ \hline
      \textbf{9x} & 60         & 81         & 4F         & DC         & 22         &
      2A         & 90         & 88         & 46         & EE         & B8         &
      14         & DE         & 5E         & 0B         & DB                        \\ \hline
      \textbf{Ax} & E0         & 32         & 3A         & 0A         & 49         &
      06         & 24         & 5C         & C2         & D3         & AC         &
      62         & 91         & 95         & E4         & 79                        \\ \hline
      \textbf{Bx} & E7         & C8         & 37         & 6D         & 8D         &
      D5         & 4E         & A9         & 6C         & 56         & F4         &
      EA         & 65         & 7A         & AE         & 08                        \\ \hline
      \textbf{Cx} & BA         & 78         & 25         & 2E         & 1C         &
      A6         & B4         & C6         & E8         & DD         & 74         &
      1F         & 4B         & BD         & 8B         & 8A                        \\ \hline
      \textbf{Dx} & 70         & 3E         & B5         & 66         & 48         &
      03         & F6         & 0E         & 61         & 35         & 57         &
      B9         & 86         & C1         & 1D         & 9E                        \\ \hline
      \textbf{Ex} & E1         & F8         & 98         & 11         & 69         &
      D9         & 8E         & 94         & 9B         & 16         & 87         &
      E9         & CE         & 55         & 28         & DF                        \\ \hline
      \textbf{Fx} & 8C         & A1         & 89         & 0D         & BF         &
      E6         & 42         & 68         & 41         & 99         & 2D         &
      0F         & B0         & 54         & BB         & 16                        \\ \hline
    \end{tabular}
  }
  \caption{S-box}
  \label{tab:cuadro4}
\end{table}
\begin{ejem}
	Si necesitamos realizar la transformación \textit{SubBytes} al número $\{19\}$, no tenemos que hacer nada mas que mirar el cuadro \ref{tab:cuadro4} , fijándonos  en la fila \textbf{1x} y la columna \textbf{x9}, así obteniendo $\{D4\}$ que es la trasformación de $\{19\}$.\\
	
\end{ejem}
\subsubsection{ShiftRows()}
Esta transformación consiste en una rotación cíclica hacia la izquierda de las
filas de la matriz de estado, de manera que la primera fila permanece igual, la
segunda fila se rota hacia la izquierda una posición, la tercera fila se rota
hacia la izquierda dos posiciones y la ultima fila se rota tres posiciones a la
izquierda.
\begin{figure}[h!]
  \includegraphics[scale=0.11]{figuras/ShiftRows.png}
  \centering
  \caption{Transformación ShiftRows}
  \label{fig: 4}
\end{figure}

\subsubsection{MixColumns()}
Esta trasformación consiste en multiplicar las columnas de bytes módulo $x^4 +
  1$ por el polinomio
\[c(x)=\{03\}x^3 + \{01\}x^2 + \{01\}x + \{02\}\]

Es mas sencillo verlo como una multiplicación matricial. Sea:

$b(x)= c(x)\otimes a(x):$
\[\begin{bmatrix}
	b_{0, c} \\
	b_{1, c} \\
	b_{2, c} \\
	b_{3, c} \\
\end{bmatrix} = \begin{bmatrix}
	02 & 03 & 01 & 01 \\
	01 & 02 & 03 & 01 \\
	01 & 01 & 02 & 03 \\
	03 & 01 & 01 & 02
\end{bmatrix} \begin{bmatrix}
	a_{0, c} \\
	a_{1, c} \\
	a_{2, c} \\
	a_{3, c} \\
\end{bmatrix} \text{para } 0 \leq c \leq \text{\textbf{Nb.}}\]

\begin{obs}
	 $\otimes$ indica la multiplicación de dos polinomios (cada uno de grado $< 4$) modulo $x^4 +1$.\\
	 \label{obs1}
\end{obs}


Los cuatro \textit{bytes} de la columna son reemplazados como se ve a
continuación:
$$b_{0, c}=(\{02\}\bullet a_{0, c})\oplus(\{03\}\bullet a_{1, c})\oplus a_{2,
      c}\oplus a_{3, c}$$
$$b_{1, 0}=a_{0, c}\oplus (\{02\}\oplus a_{1, c})\oplus(\{03\}\bullet a_{2,
    c})\oplus a_{3, c}$$
$$b_{2, c}= a_{0, c} \oplus a_{1, c} \oplus (\{02\}\bullet a_{2, c})\oplus
  (\{03\} \bullet a_{3, c})$$
$$b_{3, c}= (\{03 \}\bullet a_{0, c}) \oplus a_{1, c}\oplus a_{2, c} \oplus
  (\{02\}\bullet a_{3, c})$$

\begin{figure}[h]
  \includegraphics[scale=0.1]{figuras/MixColumns.png}
  \centering
  \caption{Transformación MixColumns}
  \label{fig: 5}
\end{figure}´

\subsubsection{AddRoundKey()}
Esta transformación consiste en aplicar la operación OR-Exclusivo (XOR) entre
cada byte de la matriz de estado que proviene de la trasformación anterior
(MixColumns) y una subclave que se genera a partir de la clave del sistema para
esa ronda.
\begin{figure}[h]
  \includegraphics[scale=0.2]{figuras/AddRoundKey.png}
  \centering
  \caption{Transformación AddRoundKey}
  \label{fig: 6}
\end{figure}

En la figura que representa el proceso de cifrado (Figura \ref{fig: 7}) esta
transformación se realiza en la ronda inicial (ronda 0). Como se definió
previamente nuestra longitud de clave será de 128 \textit{bits}, por tanto la
subclave de la ronda 0 es la propia clave de cifrado.
\begin{figure}[h]
  \includegraphics[scale=0.5]{figuras/Diagrama1.png}
  \centering
  \caption{Proceso de cifrado AES}
  \label{fig: 7}
\end{figure}

\subsection{Cifrado inverso AES}
AES es un cifrador de clave simétrica por tanto la llave que se utiliza para \textit{cifrar} se utilizará para \textit{descifrar}. \\

Básicamente, el proceso de \textit{descifrado} consiste en aplicar en orden inverso las trasformaciones aplicadas en el proceso de \textit{cifrado}.\\

Las transformaciones a aplicar son las siguientes:

\subsubsection{InvSubBytes()}
La transformación \textit{InvSubBytes} es, al igual que la transformación \textit{SubBytes}, una sustitución no lineal de bytes.\\


 \begin{table}[ht]
	\centering
	\resizebox{10cm}{!}{
		\begin{tabular}{|c|c|c|c|c|c|c|c|c|c|c|c|c|c|c|c|c|}
			\hline
			\textbf{-}  & \textbf{x0} & \textbf{x1} & \textbf{x2} & \textbf{x3} & \textbf{x4} & \textbf{x5} & \textbf{x6} & \textbf{x7} & \textbf{x8} & \textbf{x9} & \textbf{xA} & \textbf{xB} & \textbf{xC} & \textbf{xD} & \textbf{xE} & \textbf{xF} \\ \hline
			\textbf{0x} & 52         & 09         & 6A         & D5         & 30         & 36         & A5         & 38         & BF         & 40         & A3         & 9E         & 81         & F3         & D7         & FB         \\ \hline
			\textbf{1x} & 7C         & E3         & 39         & 82         & 9B         & 2F         & FF         & 87         & 34         & 8E         & 43         & 44         & C4         & DE         & E9         & CB         \\ \hline
			\textbf{2x} & 54         & 7B         & 94         & 32         & 46         & C2         & 23         & 3D         & EE         & 4C         & 95         & 0B         & 42         & FA         & C3         & 4E         \\ \hline
			\textbf{3x} & 08         & 2E         & A1         & 66         & 28         & D9         & 24         & B2         & 76         & 5B         & A2         & 49         & 6D         & 8B         & D1         & 25         \\ \hline
			\textbf{4x} & 72         & F8         & F6         & 64         & 86         & 68         & 98         & 16         & D4         & 4A         & 5C         & CC         & 5D         & 65         & B6         & 92         \\ \hline
			\textbf{5x} & 6C         & 70         & 48         & 50         & FD         & ED         & B9         & DA         & 5E         & 15         & 46         & 57         & A7         & 8D         & 9D         & 84         \\ \hline
			\textbf{6x} & 90         & D8         & AB         & 00         & 8C         & BC         & D3         & 0A         & F7         & E4         & 58         & 05         & B8         & B3         & 45         & 06         \\ \hline
			\textbf{7x} & D0         & 2C         & 1E         & 8F         & CA         & 3F         & 0F         & 02         & C1         & AF         & BD         & 03         & 01         & 13         & 8A         & 6B         \\ \hline
			\textbf{8x} & 3A         & 91         & 11         & 41         & 4F         & 67         & DC         & EA         & 97         & F2         & CF         & CE         & F0         & B4         & E6         & 73         \\ \hline
			\textbf{9x} & 96         & AC         & 74         & 22         & E7         & AD         & 35         & 85         & E2         & F9         & 37         & E8         & 1C         & 75         & DF         & 6E         \\ \hline
			\textbf{Ax} & 47         & F1         & 1A         & 71         & 1D         & 29         & C5         & 89         & 6F         & B7         & 62         & 0E         & AA         & 18         & BE         & 1B         \\ \hline
			\textbf{Bx} & FC         & 56         & 3E         & 4B         & C6         & D2         & 79         & 20         & 9A         & DB         & C0         & FE         & 78         & CD         & 5A         & F4         \\ \hline
			\textbf{Cx} & 1F         & DD         & A8         & 33         & 88         & 07         & C7         & 31         & B1         & 12         & 10         & 59         & 27         & 80         & EC         & 5F         \\ \hline
			\textbf{Dx} & 60         & 51         & 7F         & A9         & 19         & B5         & 4A         & 0D         & 2D         & E5         & 7A         & 9F         & 93         & C9         & 9C         & EF         \\ \hline
			\textbf{Ex} & A0         & E0         & 3B         & 4D         & AE         & 2A         & F5         & B0         & C8         & EB         & BB         & 3C         & 83         & 53         & 99         & 61         \\ \hline
			\textbf{Fx} & 17         & 2B         & 04         & 7E         & BA         & 77         & D6         & 26         & E1         & 69         & 14         & 63         & 55         & 21         & 0C         & 7D         \\ \hline
		\end{tabular}
	}
	\caption{InvS-box}
	\label{tab:cuadro5}
\end{table}

El cuadro \ref{tab:cuadro5} es inverso al cuadro \ref{tab:cuadro4}, que representaba la S-box, este cuadro inverso se llama InvS-Box.\\

\begin{ejem}
	Recordemos el ejemplo que se propuso en la transformación \textit{SubBytes}, donde el byte $\{19\}$ tomaba el valor $\{D4\}$. Si ahora queremos aplicar la trasnformación \textit{InvSubBytes} al byte $\{D4\}$, miramos en el cuadro \ref{tab:cuadro5} la fila \textbf{Dx} y la columna \textbf{x4}, dando como resultado el byte $\{19\}$.
\end{ejem}

\subsubsection{InvShiftRows()}
 Esta transformación consiste en una rotación cíclica de \textit{bytes} hacia la derecha. Por tanto, es una rotación en dirección opuesta a la que se propuso en la transformación \textit{ShiftRows}.
\begin{figure}[h]
	\includegraphics[scale=0.65]{figuras/InvShiftRows.png}
	\centering
	\caption{Transformación InvShiftRows}
	\label{fig: 8}
\end{figure}

\subsubsection{InvMixColumns()}
Es la operación inversa de \textit{MixColumns}, en esta transformación cada columna se trata como un polinomio y luego se multiplica módulo $x^4 + 1$ con un polinomio fijo:

\[c^{-1}(x)=\{0b\}x^3 + \{0d\}x^2 + \{09\}x + \{0e\}\]

De nuevo es más fácil verlo como una multiplicación matricial y recordando la observación \ref{obs1}, tenemos: \\

 $a(x)=c^{-1}(x) \otimes b(x)$
\[ \begin{bmatrix}
	a_{0, c}\\
	a_{1, c}\\
	a_{2, c}\\
	a_{3, c}\\
\end{bmatrix} = \begin{bmatrix}
	0e & 0b & 0d & 09 \\
	09 & 0e & 0b & 0d \\
	0d & 09 & 0e & 0b \\
	0b & 0d & 09 & 0e 
\end{bmatrix} \begin{bmatrix}
	b_{0, c}\\
	b_{1, c}\\
	b_{2, c}\\
	b_{3, c}\\
\end{bmatrix} \text{para } 0 \leq c \leq \text{\textbf{Nb.}}\] donde los \textit{bytes} son reemplazados de la siguiente manera: 
\[a_{0, c}=(\{0e\}\bullet b_{0, c}) \oplus (\{0b\}\bullet b_{1, c}) \oplus (\{0d\}\bullet b_{2, c}) \oplus (\{09\}\bullet b_{3, c})\]
\[a_{1, c}=(\{09\}\bullet b_{0, c}) \oplus (\{0e\}\bullet b_{1, c}) \oplus (\{0b\}\bullet b_{2, c}) \oplus (\{0d\}\bullet b_{3, c})\]
\[a_{2, c}=(\{0d\}\bullet b_{0, c}) \oplus (\{09\}\bullet b_{1, c}) \oplus (\{0e\}\bullet b_{2, c}) \oplus (\{0b\}\bullet b_{3, c})\]
\[a_{3, c}=(\{0b\}\bullet b_{0, c}) \oplus (\{0d\}\bullet b_{1, c}) \oplus (\{09\}\bullet b_{2, c}) \oplus (\{0e\}\bullet b_{3, c})\]

\subsubsection{AddRoundKey()} Esta transformación es la  que se describió en la Sección \textit{VI-B4}, es su propio inverso, ya que solo implica una aplicación de la operación XOR.\\

En la figura \ref{fig:9} se encuentra el diagrama del cifrado inverso de AES.
\begin{figure}[h]
	\includegraphics[scale=0.6]{figuras/diagrama2.png}
	\centering
	\caption{Proceso de cifrado inverso AES}
	\label{fig:9}
\end{figure}



\section{Resultados}
\section{Conclusiones}
\section{Trabajo a futuro}




\begin{thebibliography}{1}
  \bibitem{Will} \textsc{Stallings, W.} (2016). \textit{Cryptography and Network
    Security: Principles and Practice} (7.$^{\text{a}}$ ed.). Pearson Education.

  \bibitem{Tan} \textsc{Tanenbaum, \& Wheterall.} (2011). \textit{Redes de
    computadoras} (5.$^{\text{a}}$ ed.). Pearson Education.
  \bibitem{tap}\textsc{Tapia-Recillas, H.} (2011). \textit{Sobre algunas
    aplicaciones de los campos de Galois.} Miscelánea Matemática de la sociedad
  matemática Mexicana.
  \url{https://miscelaneamatematica.org/download/tbl_articulos.pdf2.b83a7656c55ef738.353330362e706466.pdf}
  \bibitem{NIS01} \textsc{NIST}. (2001). \textit{Announcing the Advanced
    Encryption Standard (AES)}. Federal Information Processing Standards Publication
  197.

  \bibitem{GitHub-Docs} GitHub Docs.\textit{ Repositorio proyecto de
    investigación.} \url{https://github.com/KevinHe23/Proyecto_investigacion}.

  \bibitem{GitHub-Docs1} GitHub Docs.\textit{ Repositorio implementación AES.}
  \url{https://github.com/KevinHe23/AES-python}.
\end{thebibliography}
\end{document}